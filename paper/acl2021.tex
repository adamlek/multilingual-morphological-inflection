%
% File acl2021.tex
%
%% Based on the style files for EMNLP 2020, which were
%% Based on the style files for ACL 2020, which were
%% Based on the style files for ACL 2018, NAACL 2018/19, which were
%% Based on the style files for ACL-2015, with some improvements
%%  taken from the NAACL-2016 style
%% Based on the style files for ACL-2014, which were, in turn,
%% based on ACL-2013, ACL-2012, ACL-2011, ACL-2010, ACL-IJCNLP-2009,
%% EACL-2009, IJCNLP-2008...
%% Based on the style files for EACL 2006 by 
%%e.agirre@ehu.es or Sergi.Balari@uab.es
%% and that of ACL 08 by Joakim Nivre and Noah Smith

\documentclass[11pt,a4paper]{article}
\usepackage[hyperref]{acl2021}
\usepackage{times}
\usepackage{latexsym}
\renewcommand{\UrlFont}{\ttfamily\small}
\usepackage{amsmath,amsfonts,amssymb}
\usepackage[noabbrev,capitalize]{cleveref}

% This is not strictly necessary, and may be commented out,
% but it will improve the layout of the manuscript,
% and will typically save some space.
\usepackage{microtype}

%\aclfinalcopy % Uncomment this line for the final submission
%\def\aclpaperid{***} %  Enter the acl Paper ID here

%\setlength\titlebox{5cm}
% You can expand the titlebox if you need extra space
% to show all the authors. Please do not make the titlebox
% smaller than 5cm (the original size); we will check this
% in the camera-ready version and ask you to change it back.

\newcommand\BibTeX{B\textsc{ib}\TeX}

\title{Training Strategies for Neural Multilingual Morphological Inflection}

\author{First Author \\
  Affiliation / Address line 1 \\
  Affiliation / Address line 2 \\
  Affiliation / Address line 3 \\
  \texttt{email@domain} \\\And
  Second Author \\
  Affiliation / Address line 1 \\
  Affiliation / Address line 2 \\
  Affiliation / Address line 3 \\
  \texttt{email@domain} \\}

\date{}

\begin{document}
\maketitle
\begin{abstract}
something something something something something something
something something something something something something
something something something something something something
something something something something something something
something something something something something something
something something something something something something
something something something 
\end{abstract}

\section{Introduction}

Morphological inflection is the task of transforming a lemma to its
inflected form given a set of grammatical features. 
[Transition to task...]
The task requires a model to recognize which morphological processes
should be applied to the word, such as affixing, deletion or
reduplication, 
[Typological diversity ...]
[Research problems ...]

In this paper we explore different training schemas for neural
networks, in a multilingual model for morphological inflection in 37
different languages.

We are particularly interested in how far we can get using a simple
LSTM sequence-to-sequence model with attention, augmented with
training procedures informed by human heuristics.

In particular, we consider the following training augmentations:

\begin{itemize}
\item Curriculum learning: In this module we augment the order in which the examples are presented to the model
\item Multi-task learning: We consider the formal operations required to transform a lemma into its inflected form
\item Language-wise label smooothing: We smooth the loss function when it predict a character from the correct language
\item Scheduled sampling: Rather than decoding with the gold as input, we use a probability distribution to determine whether to use the previous output of the gold as input
\end{itemize}

\section{Related work}

\section{Data}

The data released cover 37 languages of varying typology and
geneology, [stuff...]. The data for the diffrent languges vary gratly
in size, presenting a unique challange to multilingual systems.

For the low-resourced languages we use \textbf{hallucinated data}. We
follow [REF] and consider the parts of a lemma overlapping with the
inflected form as candidates for replacement.

We note that certain languages in the data use phonological symbols
and separators, we hallucinate in such a way that we dont break apart
morphemes.


\section{Method}

In this section the multilingual model and training strategies used
are presented. We employ a single model with shared parmeters
across all languages. 

\subsection{Model}

We employ a encoder-decoder architecture with attention. First we use
a LSTM to obtain a contextual representation for each word in the
lemma. We encode the tags using a self-attention module, as the order
or the tags does not matter.

The encoder has three parts, a LSTM generator and two attention
modules, one attending to the lemma and one attending to the tags. For
the lemma attention we use a content-based attention module [REF] as
it helps facilitate the copy mechanism. [Cosim attention details...]
However, only using cosim attention was shown to have pitfalls, as it
mainly focused on the most similar characters in the lemma it ignored
contextual cues relevant for the generation.

To remedy this, we combine the cosine attention with additive
attention as follows, where superscript $cos$ indicate cosine attention,
$add$ additive attention and $k$ the key:

\begin{align*}
	a^{add} & = k^\top\text{tanh}(W_ah + W_bh)\\
    a^{add} & = \text{softmax}(a^{add})\\
	att^{add} & = \sum_{t=1}^{T}a_t^{add}h_t^{add}\\
	a^{cos} & = \text{softabs}(cos(k,h))\\
	att^{cos} & = \sum_{t=1}^{T}a_t^{cos}h_t^{cos}\\
	att & = W[att^{cos}; att^{add}]
\end{align*}

We employ additive attention for the tags. In each step we pass the
concatenation of the character embedding obtained from the previous
step, the lemma attention and the tag attention to the decoder.

\begin{equation*}
\text{input}_t = [e_{t-1}; att_{char}; att_{tag}]
\end{equation*}



\subsection{Multi-task learning}

We consider the fact that strings may be transformed using operations
as an additional resource when generating inflections [REF]. 
We break down the task of applying levenshtein-distance operations
into two sub-tasks: lemma-reductions, where we predict the copy and
deletion operations and lemma-additions, where we predict the copy and
addition operations. 

For each sub-task we predict the operation based on the hidden states
generated by our neural network. In the case of lemma-reductions we
predict the operation on the hidden-states of the encoded lemma. For
lemma-additions, we predict the operations on the generated characters
from the decoder.

\subsection{Curriculum Learning}

We use the easy-first curriculum strategy [REF] to sort the data after
each epoch. For all examples in the batch we sort them according to
the loss, in ascending order such that the easy (low loss) occur
before the difficult examples (high loss).  

For the first epoch, when we dont have any loss for the examples, we
sort the dataset according to the ratio of copy levenshtein-operations
to other operations. We found that this strategy performs the best
among set of initial strategies \footnote{We experimented with: fewest
addition-operations, least-grammatical-features, and random.}

\subsection{Scheduled Sampling}

It has been shown that models trained with teacher-forcing may suffer
at inference time, due to not generating new characters given the
model output, but rather given gold characters. To address this issue
we use scheduled sampling [REF]. 

We implement a simple schedule for calculating the probability of
using the gold characters or hidden stats by using a global
sample-probability variable which is updated at each epoch. Each epoch
we decrease the initial probability of 100\% by 4\%. We sample the
probability of selecting the gold for each example every time-step.

\subsection{Training}

We use KL-divergence loss for the characters and cross-entropy loss
for both the lemma-reduction and lemma-addition tasks. Our final loss
function consists of the character generation loss, the
lemma-reduction and the lemma-addition losses summed.

\paragraph{Language-wise Label smoothing} We use language-wise label
smoothing to calculate the loss. This means that we remove $\alpha$
probability from the correct character and distribute $\alpha$
uniformly across the characters belonging to the words language. A
thing to consider is that we calculate the language membership of
characters from the training set only, so we don't want to let
$\alpha$ be to large, as some characters in the development and test
set may not be accounted for.

\paragraph{Learning rate decay with a Curriculum} The outputs will be
sorted by the diffiulty in the previous epoch, to further follow the
easy-first learning idea we employ decaying learning rate. This has
the effect that a model update its parameters \textit{more} on the
easy examples and less on difficult examples. The idea is that the
morphological processes involved in more difficult words can be
discovered from the operations involved in the easier examples.

The hyperparameters used for training are presented in Table X below.
\begin{table}[h]	
\centering
\begin{tabular}{lc}
\textsc{Hyperparameter} & \textsc{Value} \\
  \hline
  Batch Size & 256 \\
  Embedding dim & 128 \\
  Hidden dim & 256 \\
  Initial LR & 0.001 \\
  Min LR & 0.0000001 \\
  Smoothing-$\alpha$ & 2.5\% \\
\end{tabular} 
\caption{Hyperparameters.}
\label{tab:hp}
\end{table}

\section{Results}

\begin{table}[h]	
\centering
\begin{tabular}{lc}
\textsc{Language} & \textsc{Accuracy} \\
  \hline
  
\end{tabular} 
\caption{Acc}
\label{tab:accuracy}
\end{table}



\subsection{Ablation Study}

We perform an ablation study to estimate the effect of our various
additional training strategies.



\begin{table}[h]	
\centering
\begin{tabular}{lc}
\textsc{Module} & \textsc{Accuracy} \\
  \hline
  All  & \\
  Lemma-reduction & \\
  Lemma-addition & \\
  Scheduled sampling & \\
  Label smoothing & \\
  Curriculum learning & \\
\end{tabular} 
\caption{Abl}
\label{tab:abl}
\end{table}

\section{Discussion}

\section{Conclusions}


%\section*{Acknowledgments}

%The acknowledgments should go immediately before the references. Do not number the acknowledgments section.
%\textbf{Do not include this section when submitting your paper for review.}

\bibliographystyle{acl_natbib}
\bibliography{acl2021}

%\appendix



\end{document}
